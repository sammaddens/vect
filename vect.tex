% Copyright 2019-2021 by Sam Maddens

\documentclass[12pt]{article}

\usepackage[a4paper, margin = 2.5cm]{geometry}
\newcommand{\TITLE}{Vect package}
\newcommand{\SUBTITLE}{Vector notation with TikZ}
\newcommand{\AUTHOR}{Sam Maddens}
\newcommand{\VERSION}{2020/12/27 version 0.1.0}

\usepackage{fnpct}
\setfnpct{punct-after}

\usepackage[bottom]{footmisc}

\makeatletter
\renewcommand{\@makefnmark}{\hbox{\textsuperscript{{\@thefnmark}}}}
\makeatother

\usepackage[bookmarks = true, bookmarksnumbered = true, pdfpagemode = UseNone, hide links]{hyperref}
\hypersetup{pdftitle = {\TITLE{} - \VERSION}, pdfauthor = {\AUTHOR}}

\usepackage[T1]{fontenc}
\usepackage[utf8]{inputenc}
\usepackage{newtxtext}
\usepackage{newtxmath}
%\usepackage{noto-mono}
%\usepackage[sfdefault]{notomath}

\usepackage{anyfontsize}% needed to prevent font warnings caused by esvect

\usepackage{etoolbox}

\usepackage{enumitem}

\usepackage[nobottomtitles]{titlesec}

\usepackage{tikz}

\pagecolor{black!10}

\SetEnumitemKey{baseItem}{leftmargin = *, topsep = 0pt, partopsep = 0pt, itemsep = 0pt, parsep = 0pt, beginpenalty = 10000, before = {\interlinepenalty=10000}}

\newsavebox{\boxItem}

\newcommand{\tikzItem}[2][0.7ex]{%
\sbox{\boxItem}{#2}%
\tikz[baseline]{\node[inner sep = 0pt, outer sep = 0pt, anchor = center] at (0,#1) {\usebox{\boxItem}};}%
}

% Bullet
\newlist{bulItem}{itemize}{5}
\setlist[bulItem]{baseItem, label = {\tikzItem{\tikz{\fill (0,0) circle (0.175em);}}}}

\usepackage{multicol}

\raggedcolumns
\setlength{\multicolsep}{0pt}
\setlength{\columnsep}{1em}

\usepackage{array}
\newcolumntype{L}[1]{>{\raggedright\let\newline\\\arraybackslash\hspace{0pt}}p{#1}}

\newenvironment{comparision}{%
\setlength{\extrarowheight}{\medskipamount}%
\begin{tabular}{@{}l @{\hspace{1em}} @{}l @{\hspace{1em}} @{}l}%
}{%
\end{tabular}%
}

\usepackage{xparse}

\usepackage{listings}

\lstset{
breaklines = true,
basicstyle = \ttfamily\color{purple},
tabsize = 2
}

\ExplSyntaxOn
\NewDocumentCommand{\showcaseCmpA}{v}%
{\tt vect & \lstinline{#1} & \tl_rescan:nn{}{#1}}
\ExplSyntaxOff

\newcommand{\lighter}{\color{black!50}}
\ExplSyntaxOn
\NewDocumentCommand{\showcaseCmpB}{v}%
{\lighter\tt esvect & \lighter\tt#1 & \lighter\tl_rescan:nn{}{#1}}
\ExplSyntaxOff

\ExplSyntaxOn
\NewDocumentCommand{\showcase}{v}%
{\lstinline{#1} \hspace*{1em} \tl_rescan:nn{}{#1}}
\ExplSyntaxOff

\usepackage[]{vect}

\usepackage{esvect}

% Check versions of documentation and package
\makeatletter
\edef\versionVect{\csname ver@vect.sty\endcsname\space version\space\verInd@vect}
\makeatother

\ifdefstrequal{\VERSION}{\versionVect}{
	% Do nothing
}{
	\PackageWarningNoLine{}{Change \detokenize{\VERSION} to "\versionVect"}
}

% Title
\title{\TITLE\\\smallskip\large\SUBTITLE}
\author{\AUTHOR\footnote{Email: \tt sam{\color{blue}\textbf{dot}}m{\color{blue}\textbf{dot}}28{\color{blue}\textbf{at}}outlook{\color{blue}\textbf{dot}}com}
\\{\normalsize \url{https://github.com/sammaddens/vect}}}
\date{\normalsize\VERSION}

% Indent
\setlength{\parindent}{0pt}

% History log
\newenvironment{version}[2]{%
\textbf{Version #1} (#2)
\begin{bulItem}%
}{%
\end{bulItem}\bigskip%
}

\newcommand{\optPackage}[1]{{\color{blue}\tt{#1}}}
\ExplSyntaxOn
\NewDocumentCommand{\cmdPackage}{v}%
{{\color{blue}\tt#1}}
\ExplSyntaxOff

\begin{document}
\maketitle

\begin{abstract}
\noindent The \verb|vect| package provides an improved vector notation by involving \textit{TikZ}.
\end{abstract}

{\color{red}\textbf{Warning!}} The package is in a testing phase, so large changes are possible.

\tableofcontents

\newpage

\section{Introduction}

The \verb|vec| command gives a vector with the arrow notation, but it isn't designed for multiple characters. This package provides an improved vector notation by involving \textit{TikZ}. Additional, the kind of notation (arrow or bold) can easily be changed.

\section{Usage}

Load the package with:

\cmdPackage{\usepackage[<options>]{vect}}

\subsection{Package options}

The package uses by default the improved vector notation when the \verb|vec| command is called. The first two package options can change this behaviour.\bigskip

\begin{bulItem}
\item \optPackage{vecb}: Bold and upright vector notation without arrow.\bigskip

\item \optPackage{oldvec}: Preserve the original \verb|vec| command.\bigskip

\item \optPackage{frame}: Put a frame around the node and arrow when the improved vector notation is used (see section \ref{sec:frame}).
\end{bulItem}\bigskip

Multiple options can be called by separating them with a comma:

\cmdPackage{\usepackage[vecb, frame]{vect}}

\subsection{Basic usage}

Some basic commands are provided for a straightforward use.\bigskip

\begin{bulItem}
\item \cmdPackage{\vec{<arg>}}: Improved vector notation by default, depends on chosen package option.\bigskip

\item \cmdPackage{\vect{<arg>}}: Improved vector notation.\medskip

{\centering\showcase{$\vect{AB}$}\par}\bigskip

\item \cmdPackage{\vecb{<arg>}}: Bold and upright vector notation without arrow.\medskip

{\centering\showcase{$\vecb{AB}$}\par}\bigskip

\item \cmdPackage{\oldvec{<arg>}}: Original vector notation.\medskip

{\centering\showcase{$\oldvec{AB}$}\par}
\end{bulItem}\bigskip

Some examples illustrate the improved vector notation.\medskip

\begin{multicols}{2}
\showcase{$\vec{x}$}\medskip

\showcase{$\vec{A_1B}$}\medskip

\showcase{$\vec{AB_1}$}\medskip

\showcase{$\vec{A_1B_1}$}
\end{multicols}

\subsection{Index}

For properly indexing a vector, each basic command has an optional argument \textbf{after} the main argument (made possible by the \verb|xparse| package).\bigskip

\begin{bulItem}
\item \cmdPackage{\vec{<arg>}[<arg>]}: \showcase{$\vec{r}[1x]$}\bigskip

\item \cmdPackage{\vect{<arg>}[<arg>]}: \showcase{$\vect{r}[1x]$}\bigskip

\item \cmdPackage{\vecb{<arg>}[<arg>]}: \showcase{$\vecb{r}[1x]$}\bigskip

\item \cmdPackage{\oldvec{<arg>}[<arg>]}: \showcase{$\oldvec{r}[1x]$}
\end{bulItem}

\subsection{Advanced usage}

A more advanced usage can be achieved with the comma separated options (key values) in the optional argument \textbf{before} the main argument:\bigskip

{\centering\cmdPackage{\vec[<options>]{<arg>}} or \cmdPackage{\vect[<options>]{<arg>}}\par}\bigskip

The optional argument is also compatible with the other basic commands, but without effects.

\subsubsection{Frame}\label{sec:frame}

The option \verb|frame| puts a frame around the node and arrow. This is helpful for fine tuning spacing. \verb|frame| can also be used as a package option.\medskip

{
\centering\showcase{$\vec[frame]{x}[1]$}\bigskip

\scalebox{10}{$\vec[frame]{x}[1]$}\par
}

\subsubsection{Spacing}

{\color{red}\textbf{Warning!}} These options can be removed in the future.\bigskip

The following options add space around the arrow:\bigskip

\begin{bulItem}
\item \cmdPackage{left = <dim>}: \showcase{$x \vec[left = 5pt, frame]{x}[1] x$}\bigskip

\item \cmdPackage{right = <dim>}: \showcase{$x \vec[right = 5pt, frame]{x}[1] x$}
\end{bulItem}

\section{Comparison with other packages}

\subsection{Esvect}

The \verb|esvect| package\footnote{\url{https://www.ctan.org/pkg/esvect}, maintained by Eddie Saudrais} is an alternative for an improved vector notation. It's based on playing with the combination of connected lines and the arrow head. The result of the \verb|vect| package is similar, but the largest differences are:\bigskip

\begin{bulItem}
\item The rendering is smooth for all zooming factors.\nopagebreak

\verb|esvect| gives an overlap in the arrow which is visible for some zooming factors. This is caused by the used method.\bigskip

\item The index is an optional argument after the base command.\nopagebreak

\verb|estvect| needs the starred version to add an index.\bigskip

\item The last index is placed correctly.\nopagebreak

\verb|esvect| introduces an extra space before the last index.\bigskip

\item Currently, only one type of arrow head is supported.\nopagebreak

\verb|esvect| provides 8 options.
\end{bulItem}\bigskip

Some examples help to compare the two packages.

\bigskip\textbf{Arrow}\bigskip

\begin{multicols}{2}
\begin{comparision}
\showcaseCmpA{$\vec{x}$} \\
\showcaseCmpB{$\vv{x}$} \\
\end{comparision}

\begin{comparision}
\showcaseCmpA{$\vec{AB}$} \\
\showcaseCmpB{$\vv{AB}$}
\end{comparision}

\begin{comparision}
\showcaseCmpA{$\vec{\imath}$} \\
\showcaseCmpB{$\vv{\imath}$}
\end{comparision}
\end{multicols}

\bigskip\textbf{Index}\bigskip

\begin{multicols}{2}
\begin{comparision}
\showcaseCmpA{$\vec{a_{ix}}$} \\
\showcaseCmpB{$\vv{a_{ix}}$}
\end{comparision}

\begin{comparision}
\showcaseCmpA{$\vec{a}[ix]$} \\
\showcaseCmpB{$\vv*{a}{ix}$}
\end{comparision}
\end{multicols}

\bigskip\textbf{Size}\bigskip

\begin{comparision}
\showcaseCmpA{$\vec{E}_{\vec{E}_{\vec{E}}}$} \\
\showcaseCmpB{$\vv{E}_{\vv{E}_{\vv{E}}}$}
\end{comparision}\bigskip

\begin{comparision}
\showcaseCmpA{$\vec{E}^{\vec{E}^{\vec{E}}}$} \\
\showcaseCmpB{$\vv{E}^{\vv{E}^{\vv{E}}}$}
\end{comparision}\bigskip

\begin{comparision}
\showcaseCmpA{\Huge$\vec{x}$} \\
\showcaseCmpB{\Huge$\vv{x}$}
\end{comparision}

\subsection{Letterswitharrows}

Another package for vectors is \verb|letterswitharrows|\footnote{\url{https://www.ctan.org/pkg/letterswitharrows}, maintained by Max Teegen}, but its focus lies on single characters.

\section{Hacking the code}

Use the following code in the preamble if you want to change the arrow.\bigskip

\begin{lstlisting}
\makeatletter
\renewcommand{\arrowpath@vect}{%
	\fill
	% Top
	($(node@vect.north west) + (0pt - 0.5*\extraWidthMain@vect,\vOffset@vect + 1*\linewidth@vect)$)
	-- ($(node@vect.north east) + (-4.5*\linewidth@vect + 0.5\extraWidthMain@vect,\vOffset@vect + 0.5*\linewidth@vect)$)
	-- ($(node@vect.north east) + (-4.5*\linewidth@vect + 0.5\extraWidthMain@vect,\vOffset@vect + 0.5*\linewidth@vect) + (-1.25*\linewidth@vect,2.5*\linewidth@vect)$)
	% Middle
	-- ($(node@vect.north east) + (0pt + 0.5\extraWidthMain@vect,\vOffset@vect)$)
	% Bottom
	-- ($(node@vect.north east) + (-4.5*\linewidth@vect + 0.5\extraWidthMain@vect,\vOffset@vect - 0.5*\linewidth@vect) + (-1.25*\linewidth@vect,-2.5*\linewidth@vect)$)
	-- ($(node@vect.north east) + (-4.5*\linewidth@vect + 0.5\extraWidthMain@vect,\vOffset@vect - 0.5*\linewidth@vect)$)
	-- ($(node@vect.north west) + (0pt - 0.5\extraWidthMain@vect,\vOffset@vect - 1*\linewidth@vect)$)
	% Close
	-- cycle;%
}
\makeatother
\end{lstlisting}

\makeatletter
\renewcommand{\arrowpath@vect}{%
	\fill
	% Top
	($(node@vect.north west) + (0pt - 0.5*\extraWidthMain@vect,\vOffset@vect + 1*\linewidth@vect)$)
	-- ($(node@vect.north east) + (-4.5*\linewidth@vect + 0.5\extraWidthMain@vect,\vOffset@vect + 0.5*\linewidth@vect)$)
	-- ($(node@vect.north east) + (-4.5*\linewidth@vect + 0.5\extraWidthMain@vect,\vOffset@vect + 0.5*\linewidth@vect) + (-1.25*\linewidth@vect,2.5*\linewidth@vect)$)
	% Middle
	-- ($(node@vect.north east) + (0pt + 0.5\extraWidthMain@vect,\vOffset@vect)$)
	% Bottom
	-- ($(node@vect.north east) + (-4.5*\linewidth@vect + 0.5\extraWidthMain@vect,\vOffset@vect - 0.5*\linewidth@vect) + (-1.25*\linewidth@vect,-2.5*\linewidth@vect)$)
	-- ($(node@vect.north east) + (-4.5*\linewidth@vect + 0.5\extraWidthMain@vect,\vOffset@vect - 0.5*\linewidth@vect)$)
	-- ($(node@vect.north west) + (0pt - 0.5\extraWidthMain@vect,\vOffset@vect - 1*\linewidth@vect)$)
	% Close
	-- cycle;%
}
\makeatother

{\centering\scalebox{10}{$\vec[frame]{x}[1]$}\par}

\section{Improvement}

Bugs or suggestions for improvement can be reported at the related GitHub page or by email:
\begin{bulItem}
\item \url{https://github.com/sammaddens/vect}
\item \tt sam{\color{blue}\textbf{dot}}m{\color{blue}\textbf{dot}}28{\color{blue}\textbf{at}}outlook{\color{blue}\textbf{dot}}com
\end{bulItem}

\newpage

\section{History log}

\begin{multicols}{2}

\begin{version}{0.1.0}{2020/12/27}
\item Search for last index removed.
\item Code extended.
\end{version}

\begin{version}{0.0.1}{2020/03/29}
\item Code clarified and extended.
\end{version}

\begin{version}{0.0.0}{2019/07/25}
\item Initial version.
\end{version}
\end{multicols}

\end{document}